\documentclass[twocolumn]{article}
\usepackage{graphicx}
\usepackage{amsmath}
\usepackage{hyperref}
\title{Example document in \LaTeX{} about the square root function}
\author{H. H. Kristensen}
\date{\today}

\begin{document}

\maketitle

\noindent This is a quick document about the square root function. The square root function is the solution to the simple differential equation
%
\begin{equation}\label{eq:test}
y' = \frac{1}{2y}.
\end{equation}

This differential equation can for example be solved by a numerical ordinary differential equation solver, like the \emph{Runge-Kutta method}. By doing that, one can obtain data points that can be used to create a plot of the function, like the one seen in figure~\ref{fig:sqrtplot}.


\begin{figure}[h]
\input{Sqrt.tex}
\caption{Plot of the function $\sqrt{x}$. The plot has been created using Gnuplot.}
\label{fig:sqrtplot}
\end{figure}

\section{Square roots of positive integers}
The following paragraphs are taken directly from Wikipedia\cite{wiki}. A positive number has two square roots, one positive, and one negative, which are opposite to each other. So, when talking of the square root of a positive integer, this is the positive square root that is meant. 

The square roots of an integer are algebraic integers and, more specifically, quadratic integers.

The square root of a positive integer is the product of the roots of its prime factors, because the square root of a product is the product of the square roots of the factors. Since $\sqrt{p^{2k}}$ = $p^k$, only roots of those primes having an odd power in the factorization are necessary. More precisely, the square root of a prime factorization is 
%
\begin{multline}
\sqrt{p_1^{2e_1 + 1} \dots p_k^{2e_k + 1} p_{k+1}^{2e_k+1} \dots p_n^{2e_n}} \\
= p_1^{e_1} \dots p_n^{e_n} \sqrt{p_1 \dots p_k} .
\end{multline}


\section{Practical application of the square root}

The square root is a very useful function. The function makes it easy to calculate for example the distance between two points in n-dimensional Euclidean space. This can be done as
%
\begin{equation}
l = \sqrt{\Delta x_1^2 + \Delta x_2^2 \dots \Delta x_n^2}.
\end{equation}


\begin{thebibliography}{9}
\bibitem{wiki} 
Wikipedia,
\url{https://en.wikipedia.org/wiki/Square\_root}
\end{thebibliography}

\end{document}
