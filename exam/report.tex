\documentclass[twocolumn]{article}
\usepackage{amsmath}
\usepackage{hyperref}
\title{Small report on exam problem \#1 \\
\normalsize Practical programming and numerical methods \\
Exam 2020}
\author{Henrik H\o j Kristensen}
\date{\today}

\begin{document}

\maketitle

\section{Determining $b_i$, $c_i$ and $d_i$}

The cubic subspline is on the form
\begin{equation}
S_i(x) = y_i + b_i(x-x_i) + c_i(x-x_i)^2 + d_i(x-x_i)^3.
\end{equation}

The $b_i$, $c_i$ and $d_i$ coefficients are completely determined by the following three equations,
\begin{align}
S_i(x_{i+1}) =&\, y_{i+1} \label{eq:crit1} \\
S'_i(x_i) =&\, y'_i \label{eq:crit2} \\
S'_i(x_{i+1}) =&\, y'_{i+1} \label{eq:crit3}.
\end{align}

Solving these equations result in the following equations for the coefficients,
\begin{align}
b_i =&\, y'_i \\
c_i =&\, \frac{3(p_i-b_i)}{\Delta x_i} - q_i \\
d_i =&\, \frac{q_i - 2c_i}{3\Delta x_i},
\end{align}
where $\Delta x_i \doteq x_{i+1} - x_i$, $p_i \doteq (y_{i+1}-y_i)/\Delta x_i$ and $q_i \doteq (y'_{i+1}-y'_i)/\Delta x_i$.

\section{Expanding to fourth order}

To obtain a continuous second derivative, we expand the cubic subspline,
\begin{align}
S_i(x) =&\, y_i + b_i(x-x_i) + c_i(x-x_i)^2 + d_i(x-x_i)^3 \\
	+&\, e_i(x-x_i)^2(x-x_{i+1})^2.
\end{align}

Written on this form, nothing changes when we solve equation~\ref{eq:crit1}, \ref{eq:crit2} and~\ref{eq:crit3}, so all $b_i$, $c_i$ and $d_i$ coefficients stay the same.

The condition we want to fulfill to have a continuous second derivative is
\begin{equation}
S''_i(x_{i+1}) = S''_{i+1}(x_{i+1}).
\end{equation}

This results in the recursive relation for $e_i$
\begin{equation}\label{eq:eDown}
e_i = \frac{c_{i+1}-c_i - 3d_i\Delta x_i + e_{i+1}(\Delta x_{i+1})^2}{(\Delta x_i)^2}
\end{equation}

or
\begin{equation}\label{eq:eUp}
e_{i+1} = \frac{c_i - c_{i+1} + 3d_i\Delta x_i + e_i(\Delta x_i)^2}{(\Delta x_{i+1})^2}.
\end{equation}

We have to choose a starting e-coefficient before we can use the recursive relations. Let's pick $e_0 = 0$. To determine all $e_i$'s, first equation~\ref{eq:eUp} is run recursively, and then afterwards equation~\ref{eq:eDown} is run recursively backwards starting from $e_{n-1}/2$, where $n$ is the provided amount of tabulated points (this becomes $e_{n-2}/2$ in C\# due to the zero-indexing). 

\section{Remarks to BesselPlot.svg}

The integral of the zeroth spherical Bessel function from $0$ to $x$ is known as the sine integral, and turns out to be quite complicated to solve. According to Wikipedia\cite{wiki} it is known as the $Si(x)$ function,
\begin{equation}
Si(x) = \int_0^x j_0(x) dx = \int_0^x \frac{\sin(x)}{x} \mathrm{d}x.
\end{equation}
It can be written out in a long series, but it is not easy to plot, since one apparently needs to include
a long part of the series to get something that is somewhat accurate. Instead I have looked up some
table values for it via wolframalpha.com, so there is a bit of data to compare the integrated subspline expression to.

\medskip

The second derivative of the subspline is a bit wiggly. It is not precise around $x = 0$, but otherwise it follows the analytic second derivative of $j_0(x)$ fairly well. Increasing the amount of tabulated data points for $j_0(x)$ and $j'_0(x)$ will make the second derivative of the subspline follow the analytic expression very precisely (with the exception of just around $x = 0$). More tabulated data points can be generated by making the ODE solver take more steps when solving the spherical Bessel differential equation - for example by lowering the tolerated error.


\begin{thebibliography}{9}
\bibitem{wiki} 
Wikipedia,
\url{https://en.wikipedia.org/wiki/Trigonometric\_integral}
\end{thebibliography}

\end{document}
